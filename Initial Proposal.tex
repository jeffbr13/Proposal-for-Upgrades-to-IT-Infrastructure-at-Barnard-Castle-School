\documentclass[a4paper,leqno,titlepage]{article}
\usepackage[utf8]{inputenc}
\usepackage{amsmath}
\usepackage{hyperref}
\setlength{\parskip}{2ex plus 0.5ex minus 0.2ex}

\title{Proposal for Upgrades to IT Infrastructure at Barnard Castle School}
\author{The Academic Council of Barnard Castle School}
\date{2011-12-03}



\begin{document}

\maketitle

\section{Executive Summary}

This is a proposal for upgrades to the campus internet connection,
installation of wireless networking accessible to both students and staff,
and smarter ways of working with IT in the school. 

Barnard Castle School risks slipping behind in its
IT infrastructure and policy.
Without a clear and decisive plan, this risk will become a reality, putting
the school at a practical and commercial disadvantage.


For years, other schools\footnote{\href{http://www.sedberghschool.org/ict.html}{Sedbergh School}} 
% TODO: Need more examples for this, see Rev.
have been upgrading their equipment, and building extensible infrastructure
whilst we have lingered with a proprietary, inflexible, and expensive system of
computers upgraded piecemeal or not at all.


The school's current systems and infrastructure is insufficient for current
needs, and will not scale to meet the future's either.
There needs to be a greater focus on creating a more sustainable and flexible
policy that will scale with the school, it's students, and their teachers, into
the future.


We the academic council, and all the members of the student body who have signed
the attached petition, propose that the school focuses it's resources on building
a secure and flexible network environment which allows users to connect to the
internet, each other, and school resources in a controlled and efficient manner.
Such a network would include WiFi access to everyone, on their own laptops and
other equipment, and hopefully an upgrade 


By being allowed to use our own equipment, we believe that an open network will
aid communication and learning amongst the staff and students, and that by
applying less direct restrictions the experience of using a computer in the
school will have less friction and probably even cost less in the long term.


The methodology and software exists, is mature, and is well-tested. This has been
attempted before, successfully, in many other environments and schools. BCS will
not be alone in stepping into the future.


\break

\section{Rationale}


If, rather than concentrating on having enough computers for people--which we
never have, due to a substantial proportion of them being broken, in some way,
at any one time--we instead build a network that allows people to attach their
own equipment, be it a laptop, tablet, or e-reader.

Specifically we propose:
    
\begin{description}

\item[Wireless internet and network access] for everyone.
\item[School accounts] to remain for file storage, printing and email, and for
the use of the remaining school computers.
\item[Faster internet connection] for the entire campus.


\end{description}


The school currently spends a great deal of money on a patchwork of ICT
service contracts, leases, and software licences.
Much of the software we pay RM for is freely available by design, while we pay
a too much for the privilege of having their badge on commodity hardware components.
In addition, our current system requires a great deal of upkeep relative to
what we pay for it all. In buying so much of our infrastructure pre-assembled
from a third party, we have ended up with an over-complicated and
expensive system.


Boarders are unable to remain in contact with their parents nearly as much
as at other schools. If Skype and other web messaging services are essentially free, why
would they pay exorbitant fees for international phone calls? At the same time though, students
must sit in their housemaster's office to make use of Skype to have private conversations with their
relations. This is less than ideal.


Those boarders who want unfettered access to the internet are
able to gain it quite easily, using either a 3G dongle or simply walking to the NEST
café down the road. By giving them partially restricted access to the internet,
they are unlikely to spend much more energy or money trying to get at the small part
they can't access in school.


The school email system is heavily used by the staff, and is very useful, but
almost no students use it, or even know it exists. The school does take records
of students's home email addresses, but consider how we, the students,
communicate. Email accounts are relegated to mainly collect automatic notices
from other, more flexible, ways to communicate on the web, primarily an
ever-shifting miasma of 'social networks'.
The ability to store files at school is a useful one, as is accessing them from
home. But the amount of space offered is pitiful in this day and age, and it is
impossible to log on through the internet access feature.
Pupils use school computers primarily as a route to the printers, which are a
tremendously helpful resource, and should, if anything, have more money spent on them.
However, apart from that single use case,
most prefer to work at home as much as is practical.

There are various reasons for this.
Software at the school is often out of date, and therefore insecure.
Any data processed by a school computer is at greater risk than is necessary,
potentially a legal issue and ethical issue, if the school were ever to be
targeted by a malicious agent.


Internet Explorer on the computers is the primary offender: not only is it
a well-known attack vector for harmful programs and crackers, but it is slow, and
prone to crashing, taking students's work with it. Because of misconfiguration,
the homepage for most of the school is set to the security software's website,
and many don't know how to get back to the intranet page. UCAS login is
inaccessible outside of M Block because of misconfiguration, too.


This is not the only instance of the problems associated with keeping the
software on a legion of school computers coordinated. Memory Map
(used for Duke of Edinburgh's Award groups) has disappeared from every computer
in the last round of 'upgrades'. Workstations in the library cannot even open
PDF files, used for past papers and various other documents.


Equipment upkeep is poor. Outside of the M Block (or even, through no fault of
their own, the technicians' eyeline) the state of the computers quickly declines
to be near unusable. 'Warm-up' times can exceed five minutes. Ports, keyboards,
or screens might not work, or entire terminals may refuse to turn on.
Students feel little need to look after equipment that A) is not theirs,
and B) doesn't usually work anyway. The purpose of ICT is to enable
people to work and communicate faster, not to impede and distract them as it
often does now. The computer rooms under the direct jurisdiction of the
technicians are as a rule well-kept, but the space in Main School could be
better used for other purposes--a student meeting room perhaps.


Again, it is the boarders, who suffer from this the most, being unable to use
anything except the school's system to communicate. When this means the
half-dozen or so functional computers in main school being shared between the
hundred boarders, work and prep starts to be affected.
% TODO: Find out how many male boarders there are.


None of these problems are fatal, but collectively they mean that all too often,
using a school computer is akin to death by a thousand (paperless) cuts.
The complexity of dealing with all of these problems is too much for such a
small IT team to handle, and a larger one cannot be justified in a school of
this size.


To deal with this, Barnard Castle School's infrastructure can be simplified
so that it is better adapted for how it is actually used, whilst being more
readily expandable for the future.



\break

\section{Methodology}


As previously mentioned, none of the systems required are new or even
experimental. They are well-tested and used in diverse and challenging
situations.

\subsection{Faster Internet Connection}

% Short Rationale again

% Pros/Cons
This is the easiest part of the proposal to implement, and could be dropped
into the current system wholesale, with no further changes. However, as you will
already know, this is likely to be very expensive.
% TODO: Get ballpark figures


This involves the school paying to lay it's own cables to the nearest exchange
to guarantee fast access.


% Alternatives?
If the school were to approach other businesses in the local area (specifically
NEST Café) then it may be able to share the costs and the benefits of the
improved infrastructure with the local area.


% Rationale, AGAIN
The advantage of laying our own lines is that not only is it a simple upgrade
for the entire campus, but it will future-proof the school with regards to
internet access, putting us at the forefront of technology for the foreseeable
future.




\subsection{Wireless Network and Internet Access}

%TODO:



\subsection{Email, File Storage, and Printing}


%TODO:






\end{document}
