\documentclass[a4paper,leqno,titlepage]{article}
\usepackage[utf8]{inputenc}
\usepackage{amsmath}
\usepackage{hyperref}
\setlength{\parskip}{2ex plus 0.5ex minus 0.2ex}

\title{Proposal for Upgrades to IT Infrastructure at Barnard Castle School}
\author{The Academic Council of Barnard Castle School}



\begin{document}

\maketitle

\section{Executive Summary}

This is a proposal to the management of Barnard Castle School for upgrades
to the campus internet connection,
installation of wireless networking accessible to both students and staff,
and smarter ways of working with IT in the school. 

Barnard Castle School risks slipping behind in its
IT infrastructure and policy.
Without a clear and decisive plan, this risk will become a reality, putting
the school at both a practical and commercial disadvantage.


For years, other schools
\footnote{Durham Independent School}
\footnote{Pocklington School}
\footnote{Queen Elizabeth's Sixth Form College, Darlington}
\footnote{\href{http://www.sedberghschool.org/ict.html}{Sedbergh School}}
\footnote{St. Peter's School, York}
have been upgrading their equipment and building extensible infrastructure,
whilst we have lingered with a proprietary, inflexible, and expensive system of
computers upgraded piecemeal or not at all. Enough of our rivals are ahead of
us to make us seem weak in this area, but few enough for us to have a distinct
advantage if we begin to invest in our infrastructure as soon as possible.


The school's current systems and infrastructure is insufficient for present
needs, and will not scale to meet the future's either.
There needs to be a greater focus on creating a more sustainable and flexible
approach to ICT that will scale with the school, its students, and teachers,
into the future.


We, the academic council, and all the members of the student body who have signed
the attached petition, propose that the school focuses its resources on building
a secure and flexible network environment which allows users to connect to the
internet, each other, and school resources in a safe and efficient manner.
Such a network would include WiFi access to everyone, on their own laptops and
other equipment, and hopefully an upgrade to the school's internet connection.


By being allowed to use our own equipment, we believe that an open network will
aid communication and learning amongst the staff and students, and that by
applying less direct restrictions the experience of using a computer in the
school will have less friction and probably even cost less in the long term.


The methodology and software exists, is mature, and is well-tested. This has
been attempted before, successfully, in many other environments and schools.
Barnard Castle School will not be alone in stepping into the future.


\break

\section{Why This is Important}


If, rather than concentrating on having enough computers for people --- which we
never do, due to a substantial proportion of them being in some way broken
at any one time --- we instead build a network that allows people to attach their
own equipment, be it a laptop, tablet, or e-reader.

Specifically we propose:
    
\begin{description}

\item[Wireless internet and network access] for everyone.
\item[School accounts] to remain for file storage, printing and email, and for
the use of the remaining school computers.
\item[Faster internet connection] for the entire campus.


\end{description}


The school currently spends a great deal of money on a patchwork of ICT
service contracts, leases, and software licences.
Much of the software we pay RM for is freely available by design, while we pay
too much for the privilege of having their badge on commodity hardware components.
In addition, our current system requires a great deal of upkeep relative to
what we pay for it all. In buying so much of our infrastructure pre-assembled
from a third party, we have ended up with an over-complicated and
expensive solution.


Boarders are unable to remain in contact with their parents nearly as much
as at other schools. If Skype and other web messaging services are essentially free, why
would they want to pay exorbitant fees for international phone calls?
At the same time though, students must sit in their housemaster's office to make
use of Skype to have private conversations with their relations.
This is less than ideal.


Those boarders who want unfettered access to the internet are
able to gain it quite easily, using either a 3G dongle or simply visiting the NEST
café down the road. By giving them partially restricted access to the internet,
they are unlikely to spend much more energy or money trying to get at the small part
they can't access in school.


The school email system is heavily used by the staff, and is very useful, but
almost no students use it, or even know it exists. The school does take records
of students's home email addresses, but consider how we, the students,
communicate. Email accounts are relegated to mainly collect automatic notices
from other, more flexible, ways to communicate on the web, primarily an
ever-shifting miasma of social networks.


Software at the school is often out of date, and therefore insecure.
Any data processed by a school computer is at greater risk than is necessary,
potentially a legal issue and ethical issue, if the school were ever to be
targeted by a malicious agent.


Internet Explorer on the computers is the primary offender: not only is it
a well-known attack vector for harmful programs and crackers, but it is slow, and
prone to crashing, taking students's work with it. Because of misconfiguration,
the homepage for most of the school is set to the security software's website,
and many don't know how to get back to the intranet page. UCAS login is
inaccessible outside of M Block because of misconfiguration, too.


This is not the only example of the problems caused due to difficulties
associated with keeping the software on a legion of school computers
coordinated. Memory Map (used by all the Duke of Edinburgh's Award groups) has
disappeared from every computer in the last round of `upgrades'. Workstations
in the library cannot even open PDF files, used for past papers and various
other documents.

The ability to store files at school is a useful one, as is accessing them from
home. But the amount of space offered is pitiful in this day and age, and it is
impossible to log on through the internet access feature.
Pupils use school computers primarily as a route to the printers, which are a
tremendously helpful resource, and should, if anything, have more money spent on them.
However, apart from that single use case,
most prefer to work at home as much as is practical.


Equipment upkeep is poor. Outside of the M Block (or even, through no fault of
their own, the technicians's eyeline) the state of the computers quickly declines
so as to be near unusable. `Warm-up' times can exceed five minutes. Ports, keyboards,
or screens may not work, or entire terminals might refuse to turn on.
Students feel little need to look after equipment that A) is not theirs,
and B) doesn't usually work anyway. The purpose of ICT is to enable
people to work and communicate faster, not to impede and distract them as it
does now. The computer rooms under the direct jurisdiction of the
technicians are as a rule well-kept, but the space in Main School could be
better used for other purposes --- a student meeting or presentation room perhaps.


Again, it is the boarders who suffer from this the most, being being entirely
limited to the school's system to communicate. When this means the
half-dozen or so functional computers in main school being shared between the
hundred boarders, it's not only contact with home,
but work and prep which is affected.
% TODO: Find out how many male boarders there are.


None of these problems are fatal, but collectively they mean that, all too often,
using a school computer feels like death by a thousand (paperless) cuts.
The complexity of dealing with all of these problems is too much for such a
small IT team to handle, and a larger one cannot be justified in a school of
this size.


To deal with this, Barnard Castle School's infrastructure can be simplified
while better adapting it for how it is actually used, and to be
readily expandable for the future.


Barnard Castle School does not need a custom-built, from-the-ground-up,
comprehensive, or built `just for schools' computer network.
What it \emph{does} need is to make optimum use of available technologies
to make work and communication on the campus easier.


As previously mentioned, none of the technologies required are new or even
experimental. They are well-tested and used in diverse and challenging
situations.

The following parts,
`Faster Internet Connection',
`Wireless Network and Internet Access',
and `Email, File Storage, and Printing',
outline the three things we propose as necessary to bring the school IT
infrastructure into the modern age and beyond.






\section{Faster Internet Connection}


Conceptually, this is the easiest part of the proposal to implement,
and could be dropped into the current system wholesale, with no further changes.
However, it is an expensive proposition.


This involves the school paying to lay its own cables to the nearest exchange,
which will guarantee fast and almost future-proof access to the internet and
world-wide-web for years to come.


Other methods are also being looked into as well, from broadcasting a high-power
WiFi signal across from the exchange (unreliable) to patching in to another
organisation's network, although details on these plans are sketchy.


The upgrade is a simple, but expensive, one, but will set us apart from other
schools for our foresight, and belief in technological improvement.


If the school were to approach other businesses in the local area (specifically
the NEST Café, the Bowes Museum) then it may be able to share the costs
and the benefits of the improved infrastructure with the local area.


However, it should be noted that at this stage, it may no longer be particularly
advantageous to install our own infrastructure.


BT guarantees 100 Megabit/s access speeds by the end of 2014,
and Virgin are rolling out their own optic fibre network over
the entire country. Upgrades are just around the corner, and within a few years,
access speeds should not be a problem. The cost of maintaining the lower level
of access to the internet that we have now should be carefully compared to
the benefits brought about by an expensive upgrade, which will seem a lot less
necessary in a few years.


At the same time, the government has a fund set aside for local councils in
the North East to upgrade local internet connections, money which has not been
touched by anyone, anywhere. Perhaps by reaching out to Barnard Castle Town
Council, we could unlock or make use of some of this grant money.


We need to overcome the inertia on this issue, and a plan must be made and set
into motion as soon as possible if we are going to have an acceptable internet
connection in this half of the decade.






\section{Wireless Network and Internet Access}


\subsection{Rationale}

Wireless networking is something that has been promised to us as being `just
around the corner' for a number of years now.


Indeed, a school-wide wireless network is the backbone of this proposal. It is
what ties together the various other components, allowing more flexible
communications, ways of working; drawing more people to the school as a
major selling point. It will show the school's commitment to continuous
improvement and innovation, and provide a draw to parents who want their
children to be competent in the data-rich world we now live in.


It will be harder to install and maintain than a cable to the nearest exchange,
but potentially much less expensive, and will require less effort than the network
running on top of it will take, once set up.


With a WiFi network installed, the school is no longer required to keep a cohort
of computers maintained throughout the campus. Most students own computers
or netbooks. In a worst case scenario, the school could subsidise laptops for a
subsection of students, or ensure that there remains enough terminals scattered
around school for those who don't bring in their own computers.


Information literacy is becoming an ever more important factor in education,
and familiarity with computers, not just office software but basic
configuration and best practice while using them is a necessary skill.
Taking the management of all learning technology out of the
hands of students distances them from that equipment, and inhibits their
familiarity with it.


Small and medium sized businesses seldom have their own
technicians, and they operate in an increasingly wild and complex environment
of machines and software.
Computers are not going away any time soon, and if pupils
are given a chance to learn what works, and what doesn't, for themselves when
there are technicians and teachers at hand to help,
they won't be at a disadvantage in the long run, when there is no IT department
available at hand.

The main school building's construction is a very harsh environment for radio
waves to propagate. This means that a decent number of wireless access points
will be needed to give adequate coverage everywhere. However, wireless equipment
costs very little to buy off the shelf. The larger part of the cost will be
installing the cables to each access point.
On the flip-side, signal issues should allow segmentation of the building
into `zones', allowing rough-grained control, easier diagnostics, and
controlled roll-out of network access.


If the system is rolled out gradually, it can be
stress-tested by the most testing part of the student population - the boarders.
If it can be tuned according to their heavier usage patterns before expansion to
the rest of the school, then policies and minor kinks can all be worked out
before extensive investment is made.


\subsection{Installation and Hardware}

There are two routes to take in installation.
Outside specialists could be brought in, or the
school could make use of its own staff and expertise,
and endeavour to do it in-house.
Either way, a clear set of specifications are necessary to advise the network
architecture choices. The following is our suggestion for such a network.


Standard 'omnidirectional' aerials for wireless access points
actually produce a rather flat, 2D `disc' of wireless coverage, broadcasting
the signal in a radius along the dimension of orientation.
One access point per wing, on each floor, should cover the boarders's
dorms in main school, and three or four more on the ground floor to ensure
coverage to all the offices and IT rooms.


All of the access points will need to be wired up to the rest of the network
to deal with the traffic, and allow future replacement and upgrades.
Hard connections exhibit almost no lag, are almost bulletproof in
comparison to WiFi, and should be maintenance-free, once installed.


Systems do exist which allow you to `daisychain' routers together wirelessly,
so that they can operate without a hard link to the central network. However
they are usually limited to a single manufacturer's products,
and are poorly supported. Because of the way that network traffic works,
it also cuts speed in half for every step away from the central network you are.
This isn't a huge deal when surfing the internet at home,
but with more than a few network clients accessing large documents or sending
print jobs over a school network, for example,
the airwaves soon become saturated, lowering speeds for everyone.


\subsection{Content Policy}

A server currently sits directly between the unfiltered internet connection and
the rest of the campus. The school content filter and web cache is used not only
to control access to the internet, but to accelerate multiple requests for the
same resource by storing a local version, which can be returned much faster than
if it was to be sent over the internet each time. This is what keeps internet
access usable at present, when classes of pupils log on to the same website
all at once.


The caching software is sold to us directly by RM. However, as far as can can
be discerned, it is simply a customised version of a freely available web cache
\footnote{\href{http://en.wikipedia.org/wiki/Squid_(software)}{Squid open source web cache software}}.
If this has been paid for outright, then there
may be little reason to change it. But with minimal effort, we can install a
copy on a single highly specified computer to act as a a web cache which we
can control and optimise. The harder task will be setting it up to act as a
content filter as well. This process is well documented and supported though,
and will give the school greater control and smarter filtering.


% Filtration policy

The filtering server will not only be able to intercept and block banned
material, but also log attempts at accessing it. Depending on how we regulate
access to the network, there are a few solutions to this problem;
providing a network address to each individual user (easy to change,);
keeping a record of device MAC codes, the unique fingerprints for every item
authorised to be used on the network;
or simply using network credentials to log in to the service.


We are proposing a school network which is more flexible and open than before,
to allow faster and easier communication between teachers, students, and the
wider world. We on the academic council believe that this should be taken into
account by the content filtration policies.


Excessive filtration in the case of students and teachers researching sensitive
topics gets in the way of work. For boarders it means that they pay more for
contact either in the time between contact with parent, or in international
phone charges.


To invoke the access to email and static web services is to deny the reality
that these are not naturally the services naturally used by today's teenager.
Online dialogue has become so much richer and more dynamic than mere letters.
Emails are so much slower and more restrictive than instant messages, Facebook
posts, or Skype calls. Add to that the fact that most students's email accounts
are no more than bins for social network notifications anyway, and blanket
bans on these services are bound to seem more restrictive than protective.






\section{Network Services: Email, File Storage, Printing}


The school currently uses systems built by RM around standard components used
in any business. None of these need to be replaced, but it should be
considered whether they provide the best solution to our situation, as all are
easily replaceable.



\subsection{Email}

Email is provided by a Microsoft Exchange server. This is industry standard, and
should certainly be kept on. Flexible email, calendars, et.c. are integral
to staff communication within the school, the Exchange server allows access
through either a web interface or any of the plethora of devices which
can make use of Exchange's organisational capabilites.


\subsection{File Storage and User Accounts}

File Storage and user accounts are provided by the same package of software,
either a Windows Server customised by RM, or a free Linux distribution
running the SAMBA software suite, which conforms to the same specification.
RM's solution needs to be looked at in detail, as already it falls short of
being fully functional, and fit for purpose. Parts of it (web access) are
fundamentally broken. User storage needs to be accessible in as many ways
as possible, by computers over the local network, and over the web.


Storage capacity is a failing too. With the cost of hard drives
these days ($<$£50 per thousand Gigabytes), there's no need to be limited to
a hundred Megabytes per user.


A larger and more flexible space for storing data, combined with simple
guidance, means that the school can be considered a secure and backed-up
location, accessible from anywhere, safe for data.
School account can become the space to keep copies of important documents,
coursework assignments, past papers, rather than the shuffle of memory sticks
and hard drives these things are usually spread over now.


\subsection{Printing}

As mentioned before, school accounts are often considered little more than
conduits to the printers. They are the most useful resource the school
possesses. The quality and speed of equipment that can be shared inside of
school far outstrips what can be achieved by most students at home.


The school printers are already accessible over the network, and any computer
connected to it can print to one of them. All students need to know is how to
set up network printing on their own computers, and they can make use of them.

The only problem is that it is not obvious which unit is which, when viewed
blindly over the network. This can be tackled concurrently with WiFi network
installation. Each room with a printer  --- IT classrooms,
study areas, the library --- is a 'zone', likely to have its own access point
anyway. By keeping each of these as a subnet to the central network, the
printers should only be immediately visible to computers on the local network,
and in edge cases should allow connection to the device when its IP address
is known.


\subsection{Bug Reporting}

Things go wrong, equipment breaks, printers need restocking. We need a way
to communicate requests and information to the technicians.


A publicly accessible (within the school intranet) interface that allows issues
to be posted, checked off as fixed, or discussed in an open format, will
ensure that administrators are focussing on the matters most affecting users of
the network, as well as giving them a list of smaller issues to address during
periods of less activity.


Bug reporting software can be leveraged to ensure that they have constant access
to a list of potential or developing problems, so they can manage them more
effectively, or address users's concerns if they are deemed to be non-issues.


Involving the users and staff in the upkeep
of the network will give them some idea of what the technicians must
deal with, and will give them a chance to contribute to its smooth running.
Managing a network can be a difficult job, but a problem shared is a problem
halved.



\break








\section{Conclusion}

This document does not ask for the current equipment and network to be ripped
out and replaced all at once. We are not calling for mass disturbance and
upheaval simply for the sake of change.
However, we do want to see real change, from a
broken system to the kind outlined in the previous pages.


As computers fail, leases come to an end, and subscriptions expire,
take these opportunities to consolidate the management of equipment to where
upkeep can and must be guaranteed; the IT classrooms of the M Block,
and all the projectors, printers, and other peripherals
around the campus which must be kept in good condition and well supplied.


Trying to keep every aspect of a growing department in a growing school
under control is a losing game. We should pick and choose which parts
are important to control, and which can be turned into learning experiences,
for the good of both the school and the student.


However much of this treatment is considered, the most important message we
want you to take from this proposal is thus: a dialogue must be set up between
the users and administrators, so that those responsible know what the users want,
and the things which will actually help them, while the users know that real
progress is being made on their behalf; that their needs are being addressed.


This proposal, and the accompanying petition, is the most
concrete sign yet, that the students of Barnard Castle School are
frustrated with the state of IT far too often, and they \emph{do} indeed want
progress, and have been willing to, at the very least, put their name to it.


Dreaming and imagination is important, but what we want is to see progress,
and a solid, actionable plan, a timeline, \emph{anything}. Even if this means
facing that improvements won't be ourselves, but for posterity.
Continuous improvement is at the heart of any successful organisation,
and we can't afford to rest on our laurels.



\vspace{6cm}\hspace{3cm}\textbf{The Academic Council of Barnard Castle School}



\end{document}
